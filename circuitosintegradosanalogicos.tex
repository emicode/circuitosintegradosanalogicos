\documentclass{myarticle}
% Preambulo
\usepackage[utf8]{inputenc}
\usepackage[spanish]{babel}
\def\lineach#1#2{
#1\hfill #2
}
\begin{document}
\begin{center}{\huge{\bf CIRCUITOS}}\end{center}
\begin{center}{\huge{\bf ANALÓGICOS}}\end{center}
%\noindent{\LARGE{\bf CIRCUITOS INTEGRADOS ANALÓGICOS}}\par

\begin{description}
\item[{\large{\bf I.}}]{\large{\bf PROPÓSITO.}}\\
%\noindent 
PROPORCIONAR AL PERSONAL DISCENTE LOS CONOCIMIENTOS NECESARIOS PARA QUE REALICE APLICACIONES EN SISTEMAS DE INSTRUMENTACIÓN, EN FILTROS ACTIVOS, EN SISTEMAS OS\-CI\-LA\-DO\-RES, EN CIRCUITOS NO LINEALES, EN AMPLIFICADORES RE\-TROA\-LI\-MEN\-TA\-DOS, Y EN DISEÑO ANALÓGICO ASISTIDO 
POR COMPUTADORA, EVIDENCIANDO LOS VALORES FUNDAMENTALES DEL E\-JÉR\-CI\-TO Y FUERZA AÉREA MEXICANO.
\item[{\large{\bf II.}}]{\large{\bf ALCANCES.}}\\
%\noindent 
EL CONTENIDO TEMÁTICO DE LA ASIGNATURA COMPRENDE DESDE UN REPASO DE LA RESPUESTA EN FRECUENCIA DE LOS TRANSISTORES BIPOLARES, DE EFECTO DE CAMPO, HASTA UN ANÁLISIS DETALLADO DEL AMPLIFICADOR OPERACIONAL CONOCIENDO SUS LIMITACIONES Y USOS EN SISTEMAS RE\-TRO\-A\-LI\-MEN\-TA\-DOS. ESTA 
ASIGNATURA PRECEDE A ANÁLISIS DE CIRCUITOS ELÉCTRICOS.
\item[{\large{\bf III.}}]{\large{\bf  METODOLOGÍA DEL TRABAJO.}}\\
\begin{description}
\item[A.]EL PERSONAL DOCENTE APLICARÁ EL MÉTODO ACTIVO EN LA EXPOSICIÓN DE LOS TEMAS, RESOLUCIÓN DE PROBLEMAS SENCILLOS Y ASIGNACIÓN DE TAREAS QUE RE\-FUER\-CEN EL MATERIAL VISTO EN EL SALÓN DE SESIÓN.
\item[B.]EL PERSONAL DISCENTE REALIZARÁ POR LO MENOS UN TRABAJO EXTRA CLA\-SE, PREVIO A CADA UNA DE LAS EVALUACIONES PARCIALES, CON EL FIN DE REFORZAR LOS CONOCIMIENTOS TEÓRICOS Y LOS ASPÉCTOS PRÁCTICOS Y RESOLVERÁ PROBLEMAS DE DISEÑO EN LA INGENIERÍA ELECTRÓNICA, DE INSTRUMENTACIÓN Y DE COMUNICACIONES, APLICANDO PAQUETES DE CÓMPUTO PARA EL DISEÑO DE PROTOTIPOS ELECTRÓNICOS.
\item[C.]PARA COMPLEMENTAR LOS EJES TRANSVERSALES DEL CUR\-SO DE FORMACIÓN Y FOMENTAR LA EDUCACIÓN INTEGRAL DEL PERSONAL DISCENTE SE IMPARTIRÁN LOS PROGRAMAS DE APOYO EDUCATIVO, QUE TIENEN RELACIÓN CON ESTA ASIGNATURA COMO SON DESARROLLO HUMANO, DIFUSIÓN DE LA CULTURA,\ \ \ DE LA COMANDANCIA DEL CUER\-PO, TRABAJO PSICOPEDAGÓGICO, TUTORIAL, ENTRE OTROS.
\item[D.]ASIMISMO, PARA EL DESARROLLO ARMÓNICO E INTEGRAL DEL PERSONAL DISCENTE CON LA IMPARTICIÓN DE ESTA ASIGNATURA, SE PROMOVERÁN Y POTENCIARÁN LOS VALORES DE LEALTAD, HONRADEZ, HONOR, ABNEGACIÓN, ES\-PÍ\-RI\-TU DE CUERPO, PATRIOTISMO, VALOR Y DISCIPLINA EN CADA UNA DE LAS ACTIVIDADES EDUCATIVAS.
\end{description}
\item[{\large{\bf IV.}}] {\large{\bf PROCEDIMIENTOS DE EVALUACIÓN.}}\\ SE APLICARAN 3 EVALUACIONES PARCIALES Y UNA E\-VA\-LUA\-CIÓN FINAL ORDINARIA UTILIZANDO LA ESCALA DE CALIFICACIÓN DEL 0 AL 10, LA CALIFICACIÓN MÍNIMA APROBATORIA ES DE 6,CONSIDERANDO UN VALOR PARA CADA EVALUACIÓN PARCIAL ; 
LA CALIFICACIÓN FINAL SE INTEGRARÁ CON 60 $\%$ DEL PROMEDIO DE LAS EVALUACIONES PARCIALES Y EL 40$\%$ DE LA EVALUACIÓN FINAL, DE ACUERDO CON 6LO SIGUIENTE:
\begin{description}
\item[A.]EVALUACIÓN PARCIAL.
\begin{description}
\item[a.]\lineach{EXAMEN.}{70 $\%$}
\item[b.]\lineach{PARTICIPACIÓN EN SESIÓN.}{15 $\%$}
\item[c.]\lineach{TRABAJOS EXTRASESIÓN.}{15 $\%$}
\item[\ \ ]\lineach{TOTAL.}{100 $\%$}
\end{description}
\item[B.]EVALUACIÓN FINAL
\begin{description}
\item[a.]\lineach{EXAMEN ESCRITO.}{100 $\%$}
\item[\ \ ]\lineach{TOTAL.}{100 $\%$}
\end{description}
\item[C.]LA CALIFICACIÓN FINAL DE LA ASIGNATURA SE INTEGRARÁ CON:
\begin{description}
\item[a.]\lineach{PROMEDIO EVALUACIÓN PARCIAL.}{60 $\%$}
\item[b.]\lineach{EVALUACIÓN FINAL.}{40 $\%$}
\item[\ \ ]\lineach{TOTAL.}{100 $\%$}
\end{description}
\end{description}
DE ACUERDO CON EL ART. 80 PÁRRAFO V, DEL REGLAMENTO DE LA ESCUELA MILITAR 
DE INGENIEROS, QUEDA EXENTO EL PERSONAL DISCENTE; QUE OBTENGA UN PROMEDIO 
MÍNIMO DE 9.0 PUNTOS EN UNA ASIGNATURA, DESPUÉS DE HABER SUSTENTADO LOS 
EXÁMENES PARCIALES DE LA MISMA, A\-SEN\-TÁN\-DO\-SE LA CALIFICACIÓN 
OBTENIDA EN EL PROMEDIO, QUE\-DAN\-DO A ELECCIÓN DEL PERSONAL DISCENTE LA 
OPCIÓN DE PRESENTAR DICHA EVALUACIÓN, PARA INCREMENTAR SU PRO\-ME\-DIO 
GENERAL.

DE ACUERDO CON LA EVALUACIÓN CONTINUA, EN DONDE EL PERSONAL DOCENTE 
CORROBORA EL CUMPLIMIENTO DE LOS OBJETIVOS ESPECÍFICOS; ÉSTA SE REALIZARÁ 
DE ACUERDO CON EL CRITERIO DEL PERSONAL DOCENTE AL FINALIZAR LA SESIÓN, 
A TRAVÉS DE PREGUNTAS ESCRITAS U ORALES O TRABAJOS EXTRASESIÓN, QUE PERMITAN 
OBJETIVAR EL APRENDIZAJE SIGNIFICATIVO, SIENDO REGISTRADA POR EL PERSONAL 
DOCENTE PARA INCORPORAR ESTE REGISTRO A LA EVALUACIÓN SUMATORIA DEL PERSONAL 
DISCENTE.


PARA EVALUAR EL ASPECTO AXIOLÓGICO QUE SE RESALTA EN ESTA ASIGNATURA, SE 
EMPLEARÁN INSTRUMENTOS BASADOS EN LA OBSERVACIÓN, TALES COMO ESCALA 
ESTIMATIVA, LISTA DE VERIFICACIÓN O DE COTEJO, ASI COMO EL REGISTRO 
ANECDÓTICO, QUE FORMARÁN PARTE DEONTOLÓGICA DEL EJERCICIO PROFESIONAL DE 
FUTURO INGENIERO MILITAR.

CABE RESALTAR QUE EL ASPECTO AXIOLÓGICO EN ESTE CASO CARECERÁ DE VALOR 
CUANTITATIVO, POR EL CONTRARIO SU VALOR SERÁ CUALITATIVO Y SERVIRÁ COMO 
UN ME\-CA\-NIS\-MO PARA QUE EL PERSONAL DISCENTE RECIBA RETROALIMENTACIÓN POR 
PARTE DEL PERSONAL DOCENTE DE ESTA ASIGNATURA.
\item[{\large{\bf V.}}] {\large{\bf BIBLIOGRAFÍA.}}
\begin{description}
\item[A.] BÁSICA.
\begin{enumerate}
\item Malik, N. ELECTRONIC CIRCUITS; ANALISIS, SIMULATION AND DESIGN 
Prentice Hall, 1995.
\end{enumerate}
\item[B.] COMPLEMENTARIA.
\begin{enumerate}
\item Sedra A.S., y Smith K.C. MICROELECTRONIC CIRCUITS SAUNDERS COLLEGE 
PUBLISHING, 3a. edición, 1991. 
\item Savant S.J., Roden M.S. y Carpenter G. ELECTRONIC DESIGN, CIRCUITS 
AND SYSTEMS. 2/a. Ed. B. Cummings Pub. Co. 1991.
\item J. Millman MICROELECTRONICS McGraw Hill Books Co., 2a. edición, 1989.
\item Schilling D. y Belove C. ELECTRONICS CIRCUITS, DISCRETE AND INTEGRATED 
McGraw Hill Books Co. 3a. edición, 1989.
\item D.G.E.M. Y RECTORÍA DE LA U.D.E.F.A., ÉTICA Y MORAL MILITAR EN EL 
EJÉRCITO Y FUERZA AÉREA MEXICANOS., ESCUELA DE PENSAMIENTO MILITAR., 2010.
\item MANUAL DE ÉTICA, VALORES Y VIRTUDES MILITARES
\item PROGRAMA DE CAPACITACIÓN Y SENSIBILIZACIÓN PARA EFECTIVOS EN 
PERSPECTIVA DE GÉNERO 2008-2011
\end{enumerate}
\end{description}
\end{description}
\eject
\begin{center}
{\bf OBJETIVO GENERAL.}
\end{center}

\noindent
AL TÉRMINO DE LA ASIGNATURA, EL PERSONAL DISCENTE SERA CAPAZ DE DISEÑAR CIRCUITOS LINEALES, NO LINEALES, OS\-CI\-LA\-DO\-RES Y DE TIEMPO USANDO AMPLIFICADORES OPERACIONALES, EVIDENCIANDO LOS VALORES FUNDAMENTALES DEL EJÉRCITO Y FUERZA AÉREA MEXICANO.
\tableofcontents
\section{ EL AMPLIFICADOR OPERACIONAL COMO CIRCUITO INTEGRADO.}
\subsection*{El amplificador diferencial.}
\subsection*{Los espejos de corriente.}
\subsubsection*{Etapas de amplificación intermedias y de salida.}
\subsection*{Impedancias de entrada y salida.}
\subsection*{Polarización.}
\subsection*{Amplificadores operacionales típicos.}
\section{ EL AMPLIFICADOR OPERACIONAL IDEAL.}
\subsection*{Características del amplificador operacional ideal.}
\subsection*{Configuraciones básicas con retroalimentación: inversora, no-inversora, seguidora de voltaje, diferencial y de instrumentación.}
\subsection*{Circuitos lineales para suma, resta y solución de ecuaciones.}
\subsection*{Circuitos convertidores de voltaje a corriente y de corriente a voltaje.}
\subsection*{Circuitos para computación analógica: integradores y diferenciadores.}
\subsection*{Circuitos no lineales con diodos y transistores: amplificadores logarítmicos y antilogarítmicos.}
\subsection*{\bf PRIMER EXAMEN PARCIAL.}
\subsection*{\bf REVISIÓN DE LA EVALUACIÓN}
\subsection*{Amplificadores de aislamiento.}
\subsection*{Circuitos limitadores con retroalimentación.}
\subsection*{Circuitos comparadores.}
\subsection*{Circuitos de histéresis ``Schmitt Triggers'').}
\subsection*{Circuitos convertidores de voltajea frecuencia y de frecuencia a voltaje (V-F y F-V).}
\subsection*{Circuitos convertidores análogo-digital y digital-análogo (A-D y D-A).}
\section{ FILTROS ACTIVOS.}
\subsection*{Filtros activos modulares.}
\subsection*{Filtros tipo Chevychev y Butterworth.}
\subsection*{Otras configuraciones especiales.}
\subsection*{\bf SEGUNDO EXAMEN PARCIAL.}
\subsection*{{\bf REVISIÓN DE LA EVALUACIÓN.}}
\section{ EL AMPLIFICADOR OPERACIONAL REAL.}
\subsection*{Respuesta a la frecuencia y amplificadores retroalimentados.}
\subsection*{Impedancia de entrada y salida.}
\subsection*{Razón de rechazo de modo común (CMRR).}
\subsection{Voltaje de desvío (``offset'').}
\subsection*{Rapidez de respuesta ``Slew Rate'').}
\subsection*{Comparación de parámetros en los amplificadores operacionales reales, compensados y no compensados.}
\section{ OSCILADORES.}
\subsection*{Oscilador Colpitts.}
\subsection*{Oscilador puente de Viena.}
\subsection*{Oscilador de desviación de fase.}
\subsection*{Oscilador de cristal.}
\subsection*{Circuitos temporizadores para electrónica digital.}
\subsection*{{\bf TERCER EXAMEN PARCIAL.}}
\subsection*{{\bf REVISIÓN DE LA EVALUACIÓN}}
\subsection*{{\bf EXAMEN FINAL.}}
\end{document}
